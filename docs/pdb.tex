\documentclass[11pt]{article}
\usepackage{codespelunking}
\usepackage[pdftex]{hyperref}
\title{Packet Debugger}
\author{George V. Neville-Neil}
\begin{document}
\maketitle

\section{Introduction}

The Packet Debugger, \emph{pdb} is a program which allows people to
work with packet streams as if they were working with a source code
debugger. Users can list, inspect, modify, and retransmit any packet
from captured files as well as work with live packet capture.

Installing \program{pdb} is covered in the text file,
\file{INSTALLATION}, which came with this package.  The code is under
a BSD License and can be found in the file \file{COPYRIGHT} in the
root of this package.

\section{A Quick Tour}

For the impatient this section is a 5 minute intro to using the packet
debugger.

\section{Starting \program{pdb}}

\section{Command Reference}

\subsection{help}

\begin{verbatim}
pdb> help

Documented commands (type help <topic>):
========================================
break     create  info  next  print  run   set 
continue  delete  list  prev  quit   send  show

Undocumented commands:
======================
help  update
\end{verbatim}

The help command prints out the available topics for help.  To get
help on a specific command type \verb|help command| where
\verb|command| is one of the commands listed when you ask for help on
its own.

\subsection{quit}

\begin{verbatim}
pdb> quit
Bye
localhost ? 
\end{verbatim}

Quit the program.  If you have modified any streams you will asked if
you wish to save them before exiting.

\subsection{Loading and Saving Streams}

Each of the commands in this section works on a stream, which is the
basic unit on which pdb operates.

\subsubsection{create}

Read a new stream from a file, or open a network connection.
Currently only \program{pcap} files are supported by the \verb|create|
command.  

\subsubsection{delete}

\subsubsection{info}

\subsection{Inspecting a Stream}

\subsubsection{list}

\subsubsection{print}

\subsubsection{next}

\subsubsection{prev}

\subsection{Running a Strream}

\subsubsection{run}

\subsubsection{break}

\subsubsection{continue}

\subsection{Working with Packets}

\subsubsection{send}

\subsubsection{update}

\subsection{Debugger Options}

\subsubsection{show}

\subsubsection{set}

\end{document}
\begin{thebibliography}{99} 
\end{thebibliography}
